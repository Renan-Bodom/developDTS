\textbf{Uso de tecnologia}\\

\underline{1 - Papel estratégico da TI}\\

\textit{Contextualização}

Algumas organizações podem naturalmente terem propensão a se transformarem digitalmente, seja pela área de atuação, ou o modelo de negócio que permite grandes modificações. Neste sentido, classifique o papel estratégico que o TI terá na sua organização, podendo ser ela facilitadora ou apoiadora:

- Facilitador: em algumas organizações é necessário um impulso inicial com o uso de uma nova tecnologia digital, desta maneira, facilitador é quando a transformação digital não ocorre naturalmente, sendo necessário um facilitador.

- Apoiar: em outras organizações, as transformações ocorrem naturalmente, e o uso de uma nova tecnologia tem o intuito de apoiar.\\

\textit{Pergunta}

Qual a importância que o TI terá em sua organização, para atingir os objetivos da transformação digital? Ou descreva outro papel que o TI terá na organização.

( ) Facilitador

( ) Apoiar

Outro:\\


\underline{2 - Quão ambiciosa sua organização é diante de novas tecnologias?}\\

\textit{Contextualização}

A ambição tecnológica está ligada com o quão inovadora, ou conservadora será sua organização diante das tecnologias a serem adotadas.

- Inovador: quando a organização quer ser autora de ideias inovadoras.

- Early adopter: quando a organização quer ser a primeira a testar e fornecerem comentários de novas tecnologias.

- Seguidor: quando se deseja ser mais conservador, e utilizar tecnologias já estabelecidas.\\

\textit{Pergunta}

Qual a ambição tecnológica será buscada pela sua organização? Ou descreva outra ambição.

( ) Inovador

( ) Early adopter

( ) Seguidor

Outro:\\

\textbf{Mudanças na criação de valor}\\

\underline{3 - Grau de diversificação digital?}\\

\textit{Contextualização}

A diversificação digital é como a organização pretende disponibilizar os produtos da organização, visando produtos atuais e produtos a serem disponibilizados ao decorrer da transformação digital:

- Canais de vendas eletrônicos: Mantendo produtos analógicos, apenas utilizando um canal para alcançar clientes.

- Mídia cruzada: Utiliza soluções digitais para armazenamento de dados.

- Mídia enriquecida: Armazenamento aprimorado, possibilitando por exemplos ferramentas de buscas.

- Plataformas de conteúdo: Possibilita trabalhar com os dados, como, por exemplo, minerar dados.

- Negócio ampliado: Possibilidade de criar valores organizacional e ampliando os produtos, por exemplo, gerando informações extras.

- Organização sem fins lucrativos: Quando a organização não tem o objetivo de gerar receita.\\


\textit{Pergunta}

Como sua organização pretende impulsionar as vendas digitais? Ou descreva outra diversificação.

( ) Canais de vendas eletrônicos

( ) Mídia cruzada

( ) Mídia enriquecida

( ) Plataformas de conteúdo

( ) Negócio ampliado

( ) Organização sem fins lucrativos

Outro(s):\\


\underline{4 - Criação de receita}\\

\textit{Contextualização}

A criação de receita tem o objetivo de definir meio para que a organização se mantenha financeiramente:

Utilizar soluções digitais, podem acarretar gastos adicionais. Desta maneira, a organização precisa definir maneiras para se manter financeiramente.


- Conteúdo pago: necessário pagamento para clientes acessarem o site.

- Freemium: conteúdos adicionais ou parte do site apenas para assinantes.

- Propaganda: site exibindo propagandas.

- Produtos complementares: Serviços/produtos extras, além do principal.\\


\textit{Pergunta}

Como a organização vai adaptar os produtos para gerar receita? Ou descreva outra adaptação.

( ) Conteúdo pago

( ) Freemium

( ) Propaganda

( ) Produtos complementares

Outro(s):\\

\underline{5 - Futuro escopo do negócio principal}\\

\textit{Contextualização}

A transformação digital altera o modelo de negócio, e em função aos artefatos da organização, o escopo do negócio principal precisa ser definido. Organizações podem desde gerar novos conteúdos, até mesmo fornecer uma plataforma para outras organizações.

- Criação de conteúdo: criar conteúdos novos.

- Agregação de conteúdo: juntar/concentrar conteúdos.

- Distribuição de conteúdo: fornecer suporte para distribuir o conteúdo.

- Gerenciamento de plataformas de conteúdo: fornecer suporte para a plataforma de conteúdo, para conseguir gerenciar os conteúdos.\\


\textit{Pergunta}

Qual o futuro escopo do negócio principal? Ou descreva outro escopo.

( ) Criação de conteúdo

( ) Agregação de conteúdo

( ) Distribuição de conteúdo

( ) Gerenciamento de plataformas de conteúdo

Outro:\\


\underline{6- Eficiência em sua rede de valor}\\

\textit{Contextualização}

Os meios digitais vêm facilitando as conexões entre organizações, possibilitando criar uma rede que impulsiona os valores da organização. Adotar algumas delas podem ser benéfico.

- Parcerias: Fazer parte de um grupo de soluções, que de maneira remota simplifica o processamento e comunicações de dados.

- Terceirização: Permite uma alocação eficiente de recurso, permitindo a produção de bens selecionados, etapas de produção preliminares ou focar em processo específico. 

- Intercâmbio eletrônico de dados: Também conhecido como EDI, é um sistema eletrônico de troca de informações virtual entre empresas. Utilizado, por exemplo, para enviar documentos, notas fiscais, recibos de pagamento e outros.\\


\textit{Pergunta}

Como sua organização pode alavancar a eficiência em sua rede de valor? Ou descreva outra maneira.

( ) Parcerias

( ) Terceirização na produção

( ) Intercâmbio eletrônico de dados

Outro(s):\\


\underline{7 - Onde não envolver transformação digital}\\

\textit{Contextualização}

Canais de comunicação e implementação de tecnologia de produção são impressíveis na transformação digital. No entanto, alguns itens podem ser peças-chave para o modelo de negócio da organização e não podem ser alterados.\\


\textit{Pergunta}

Onde é uma opção não se envolver na transformação digital? Ou descreva outra(s) opções.

( ) Processamento de matéria-prima

( ) Atendimento ao cliente

Outro(s):\\

\textbf{Mudanças estruturais}\\

\underline{8 - Responsabilidade pela estratégia de transformação digital?}\\

\textit{Contextualização}

O responsável pela estratégia de transformação digital deve acompanhar a execução das próximas atividades, assim como também favorecer o uso das instruções para elaborar uma estratégia de transformação digital.

- CEO do grupo: O chefe do grupo Diretor Executivo

- CEO da unidade de negócios: O CEO da unidade de negócios que lida com o empreendimento de TD.

- CDO do grupo: Vale tanto para diretor de diversidade (responsável por ações de inclusão social) quanto para diretor de projeto.

- CIO do grupo: Responsável pelos assuntos relacionados à informática nas empresas.\\


\textit{Pergunta}

Quem será o responsável pela estratégia de transformação digital na sua organização?

( ) CEO do Grupo

( ) CEO da unidade de negócios

( ) CDO do Grupo

( ) CIO do Grupo

Outro:\\

\underline{9 - Posicionamento organizacional de novas atividades?}\\

\textit{Contextualização}

A transformação digital favorece a criação de novas atividades, assim como modificações em atividades existentes. Neste sentido, a organização deve definir se vai incorporar novas atividades de maneira integrada, ou separada:

- Integrado: novas atividades ocorrera no mesmo ambiente das antigas atividades.

- Separados: novas atividades ocorrera em um ambiente separado, não impactando nas atividades tradicionais.\\


\textit{Pergunta}

Selecione como a novas atividades serão incorporadas na organização? Ou descreva outra maneira.

( ) Integrado

( ) Separados

Outro:\\


\underline{10 - Foco das mudanças operacionais?}\\

\textit{Contextualização}

O foco das mudanças operacionais tem o objetivo de definir qual será o foco operacional após a transformação digital:


- Produtos e serviços: foca em produtos e serviços.

- Processos de negócios: visando melhorar o processo de negócio, por exemplo, agilizando ou melhorando.

- Habilidades: desenvolver novas atividades na organização.\\


\textit{Pergunta}

Que categorias de mudanças operacionais você espera?

( ) Produtos e serviços

( ) Processos de negócios

( ) Habilidades

Outra(s):\\

\underline{11 - Construção de competências}\\

\textit{Contextualização}

Com a implementação de novas tecnologias, soluções e atividades, novas habilidades são necessárias. Desta maneira, a organização precisa definir como as novas competências serão supridas:

- Internamente: contar com as competências que já tem na organização.

- Parcerias: conhecimentos fornecidos por parceiros.

- Aquisições: acumular competências.

- Fonte externa: atrair funcionários "nativos digitais", que não necessariamente precisa de um treinamento, mas que saibam avançar.\\


\textit{Pergunta}

Como sua organização pode concretizar uma estrutura de habilidades entre os funcionários? Ou descreva outros meios.

( ) Internamente

( ) Parcerias

( ) Aquisições

( ) Fonte externa

Outro(s):\\


\underline{11 - Construção de competências}\\

\textit{Contextualização}

Com a implementação de novas tecnologias, soluções e atividades, novas habilidades são necessárias. Desta maneira, a organização precisa definir como as novas competências serão supridas:

- Internamente: contar com as competências que já tem na organização.

- Parcerias: conhecimentos fornecidos por parceiros.

- Aquisições: acumular competências.

- Fonte externa: atrair funcionários "nativos digitais", que não necessariamente precisa de um treinamento, mas que saibam avançar.\\


\textit{Pergunta}

Como sua organização pode concretizar uma estrutura de habilidades entre os funcionários? Ou descreva outros meios.

( ) Internamente

( ) Parcerias

( ) Aquisições

( ) Fonte externa

Outro:\\


\underline{12 - Competências e inspirações necessárias}\\

\textit{Contextualização}

- Serviços de consultoria: Aconselhamento especializado, realizado por especialistas, que orienta para atingir seus objetivos.

- Feiras comerciais: Exposição de produtos e maquinários, que ampliando as fontes de criação de valor, além de atender clientes e destacar a própria organização.

- Meios acadêmicos: Dedicar intensamente, com muito estudo e pesquisa, para colher os frutos.\\


\textit{Pergunta}

Como novas competências e inspiração necessárias podem ser adquiridas? Ou descreva outras opções.

( ) Serviços de consultoria

( ) Feiras comerciais

( ) Meios acadêmicos

Outro(s):\\


\textbf{Aspectos financeiros}\\

\underline{13 - Pressão financeira no negócio principal?}\\

\textit{Contextualização}

Dependendo do grau de transformação, recursos financeiros podem ser mais exigidos, por isso o grau de resistência financeira deve ser esclarecida. Para que assim seja criada uma estratégia condizente com a realidade financeira disponível.

- Baixo: baixa resistência financeira.

- Médio: resistência financeira média.

- Alta: alta resistência financeira.\\


\textit{Pergunta}

Quão forte é a pressão financeira no negócio principal atual?

( ) Baixo

( ) Médio

( ) Alto\\


\underline{14 - Financiamento de novas atividades?}\\

\textit{Contextualização}

Parcerias podem ser criadas para a execução de novas atividades, por isso, é importante definir se a organização está aberta para receber financiamento externo, ou se todo financiamento será de maneira interna.

- Interno: a própria organização esta apta em financiar novas atividades.

- Externa: a organização necessita de patrocinadores.\\


\textit{Pergunta}

Como a organização financiará a realização da transformação digital?

( ) Interno

( ) Externo