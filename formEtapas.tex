\textbf{Etapa 1 - Criação da visão e objetivos da transformação digital}\\

\textit{Contextualização}

Esta etapa consiste em definir metas de longo prazo (cerca de 5 anos), é preciso alcançar uma visão global do futuro.
Em primeiro lugar, é necessário definir objetivos e metas que deseja alcançar por meio da transformação digital, implicando em seus negócios e na experiência que deseja alcançar com seus clientes e funcionários. No entanto, deve-se lidar com a realidade, deste modo, objetivos devem ser definidos com recurso de curto e longo prazo. Em segundo lugar, se concentrar em vantagens competitivas, identificando lacunas na estrutura atual e possíveis implementações para melhorar.\\

\textit{Atividades}
\begin{itemize}
    \item Liste os objetivos e visões que deseja alcançar com clientes por meio da transformação digital.
    
    \item Liste os objetivos e visões que deseja alcançar com funcionários por meio da transformação digital.
    
    \item Liste lacunas na estrutura atual.
\end{itemize}


\textbf{Etapa 2 - Avalie a capacidade de transformação digital da organização}

\textit{Contextualização}

Esta etapa consiste em avaliar a infraestrutura e o quão bem o sistema, software e ferramentas tratam do presente e do que será necessário no futuro da organização. Esta avaliação visa descobrir qual tecnologia deve ser atualizada, processos que precisam ser automatizados ou otimizados e definir ferramentas que devem ser alteradas.\\

\textit{Atividades}
\begin{itemize}
    \item Descreva a estrutura atual da organização, citando sistemas, software e ferramentas utilizadas.

    \item Conforme os objetivos descritos na etapa anterior, descreva itens como sistemas, software ou ferramentas que ainda não possui, e são necessários para atingir os objetivos.
\end{itemize}

\textbf{Etapa 3 - Projete a experiência do usuário final e do funcionário}\\

\textit{Contextualização}

Esta etapa consiste em desenvolver metas para simplificar o trabalho dos funcionários e facilitar o acesso através de novos aplicativos, funções ou sistemas, além de melhorar o acesso dos clientes. Deve-se concentrar na experiência que deseja ter com funcionários e clientes, e não em novas soluções ou limitações atuais.\\

\textit{Atividades}
\begin{itemize}
    \item Descreva metas para novos aplicativos, funções ou sistemas que devem simplificar o trabalho dos funcionários.

    \item Descreva metas para melhorar o acesso dos clientes com as novas ferramentas.

    \item Descreva soluções já existentes, que podem ser mantidas ou que precisam de poucos ajustes.
\end{itemize}

\textbf{Etapa 4 - Revisar, selecionar soluções e fornecedores.}\\

\textit{Contextualização}

Esta etapa consiste em avaliar soluções candidatas e ofertas de diferentes fornecedores de tecnologia. Validando e selecionando as soluções candidatas para atender aos objetivos desenvolvidos, entregar a experiência e fechar as lacunas das tecnologias atuais.\\

\textit{Atividades}
\begin{itemize}
    \item Descreva possíveis soluções e/ou serviços de fornecedores para atender aos objetivos, experiências e fechar lacunas descrita nas etapas anteriores.

    \item Para cada solução e/ou serviço descrito, descreva possíveis fornecedores.

    \item Descreve de que maneira novas soluções podem ser financiadas.
\end{itemize}